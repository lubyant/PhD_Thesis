\chapter{Conclusions and Future Work}
\label{chapter6}

\section{Conclusions}
\label{final_Conclusions}

Coastal zones are among the most densely populated and economically developed
regions in the world. This is equally true for Lake Michigan, where major urban
centers such as Chicago and Milwaukee are located along its shoreline. At the
same time, these regions are highly dynamic, making the prediction and
assessment of their impacts on communities and infrastructure particularly
challenging. A prominent hazard in these settings is coastal geomorphological
change, one of the most pressing concerns due to its potential to damage
infrastructure and threaten human safety. To address this challenge, numerous
studies have investigated the drivers and mechanisms of coastal change,
including water-level fluctuations, longshore sediment transport, and human
interventions. However, the role of waves—especially from the perspective of
their directional and spectral characteristics—has often been underexplored in
Lake Michigan. This dissertation provides a comprehensive assessment of coastal
geomorphological changes and wave climates in Lake Michigan, as well as their
interconnections. The research is organized into four major components. Chapter
\ref{Chapter2} presents the patterns and characteristics of coastal
geomorphological change in Lake Michigan. Chapter \ref{chapter3} offers a
detailed characterization of the directional wave climate. Chapter
\ref{chapter4} further examines wave climate from the perspective of wave
systems. Finally, Chapter \ref{chapter5} integrates these findings through the
development of a novel indexing framework—the Coastal Vulnerability Index—to
highlight the links between wave climate and geomorphological change.

Chapter \ref{Chapter2} lays the foundation for this study by characterizing the
coastal geomorphology and quantifying geomorphological changes along the
southwestern coast of Lake Michigan. This analysis incorporates multiple aspects
of coastal change, including bluff toe and crest positions, shoreline location,
beach width, bluff slope, and the presence of coastal structures. Moving beyond
a conventional erosion-focused study, this chapter introduces a multiscale
framework, evaluating changes across three spatial scales and two temporal
scales. The outcomes not only provide a publicly available dataset of local
coastal geomorphological change but also demonstrate the value of multiscale
analysis as a practical tool for coastal management. By offering flexible and
cost-effective information, this approach has the potential to enhance
decision-making for coastal managers.

Chapter \ref{chapter3} examines the wave climate at the study sites, with a
particular emphasis on wave directionality. In this chapter, a novel
metric—Directional Wave Entropy (DWE)—is introduced to rigorously characterize
the bi-directional nature of the wave climate. Using this approach,
intra-annual, inter-annual, and extreme directional patterns are systematically
evaluated. The findings demonstrate that bi-directional wave climates vary
significantly across temporal scales and directional components. These results
address the research question outlined in Table \ref{overview}, underscoring
that wave directionality is a critical aspect of wave climate, with direct
implications for coastal erosion processes and the design of coastal structures.

Chapter \ref{chapter4} investigates the wave system perspective of wave climate.
Wave spectra are constructed, and spectral partitioning is applied to extract
the directional characteristics of wave systems in Lake Michigan. Analysis of
data from two offshore buoys reveals distinct patterns between wind-sea waves
and swell, directly addressing the research question outlined in Table
\ref{overview}. In addition, nearshore wave system climates are examined and
compared with the coastal geomorphological changes presented in Chapter
\ref{Chapter2}. This work fills a critical knowledge gap by providing the first
comprehensive assessment of the spectral wave climate in Lake Michigan.


Chapter \ref{chapter5} synthesizes the findings from the preceding chapters. In
this chapter, the Coastal Vulnerability Index (CVI) is developed to evaluate the
susceptibility of shorelines to erosion, building on the geomorphological
changes documented in Chapter \ref{Chapter2}. The CVI is specifically designed
to account for Lake Michigan’s unique wave climate, incorporating both the
bi-directional characteristics identified in Chapter \ref{chapter3} and the
spectral properties described in Chapter \ref{chapter4}. Evaluation of the CVI
highlights both its limitations and its potential applications, particularly for
headland beaches influenced by bi-directional wave climates. This framework
offers a valuable tool to inform coastal management and decision-making for
communities along the Lake Michigan shoreline.

\section{Future Work}
\label{final_future_work}

While future directions have been outlined within individual chapters (\eg
Section \ref{limitations of CGC analysis}, Section \ref{c3_Limitations of DWE
and future applications}, Section \ref{c4_Limitation}, Section \ref{Limitation
of CVI}), several broader research avenues emerge from this dissertation: \\
\textbf{Coastal geomorphological changes in the Great Lakes}  

Chapter \ref{Chapter2} presented a regional-scale (approximately 100 km)
analysis of coastal geomorphological change along the southwestern coast of Lake
Michigan. Future research should extend this scope to larger spatial scales,
such as lake-wide or even Great Lakes–wide investigations, to provide a more
comprehensive understanding of coastal status at state or regional levels. Such
broader analyses would also help advance the study of nearshore processes,
particularly longshore sediment transport, across multiple spatial contexts.\\
\textbf{AI-assisted studying of coastal geomorphology}  

Expanding the spatial and temporal scope of geomorphological studies inevitably
increases the demand for human labor in coastal mapping. The integration of
artificial intelligence techniques offers an efficient alternative, enabling
automated characterization, monitoring, and assessment of coastal changes.
Approaches leveraging neural networks and large language models hold particular
promise for reducing manual effort and improving the scalability of coastal
research and management.\\
\textbf{Great Lakes wave climate versus oceanic wave climate}  

Chapters \ref{chapter3} and \ref{chapter4} investigated the directional and
system-based aspects of Lake Michigan’s wave climate. A natural extension of
this work is to examine how Great Lakes wave climates differ from their oceanic
counterparts, particularly with respect to wave directionality and wave system
dynamics. Addressing this question would provide valuable insights for refining
coastal management strategies specific to the Great Lakes region.  
