\chapter{Introduction}
\label{ch:Introduction}

\section{Background}
\label{sec:Background}

\subsection{Coastal geomorphology}
\label{subsec:Coastal geomorphology}

Coastal geomorphologies in Lake Michigan are diverse and include shorelines,
beaches, bluffs, dunes, and other landforms \citep{jackson_coastal_2013}. Two of
the most common landforms among them are shoreline and bluff, which are shown in
Figure \ref{fig:fig1.1}a. Shoreline is the natural boundary of water body and
land \citep{del_rio_error_2013}, excluding areas modified by human structures
such as revetments and ripraps. Natural shorelines are the mostly commonly found
along the coast of Lake Michigan, comprising over 80 percent of the total
coastline (see Figure \ref{fig:fig1.1}b), with the exception of certain
urbanized areas such as Chicago and Milwaukee. The second common coastal feature
is glacial till bluff, a vertical landform which is formed by glacial movement
\citep{mickelson1977shoreline,mickelson2004erosion}. Bluffs typically rise
behind the shoreline and beach, with heights ranging from one to forty meters
\citep{mickelson2004erosion}. In Lake Michigan, bluffs are primarily distributed
along the western and eastern coasts, as shown in Figure \ref{fig:fig1.1}c,
accounting for approximately 40 percent of the total lake coastline. Both
shorelines and bluffs are not only common but also significant, as they are
located in or near populated and developed areas such as Chicago and Milwaukee.
These areas affect the lives of over 10 million people and protect properties
valued at more than 1,000 billion dollars, with an estimated cost of $ \$10,000$
per linear foot of lakefront \citep{folger_coastal_1996}. Overall, coastal
geomorphologies, especially coastal bluff and shoreline, are both prevalent and
important in Lake Michigan coastal environments, necessitating comprehensive and
detailed studies.
\begin{figure}[htbp]
  \centering
  \includegraphics[width=0.8\textwidth]{chapter1/resources/figure1-1.jpg}
  \caption{The common coastal geomorphologies in Lake Michigan. (a) oblique photos of bluff and shoreline (bluff and shoreline), (b) natural shoreline map and (c) bluff map (tall bluffs over 10 meter) in Lake Michigan.}
  \label{fig:fig1.1}
\end{figure}

Coastal geomorphic change (CGC), also known as coastal accretion and erosion,
are the evolution and transformation of coastal geomorphology over time
\citep{biswas_effects_2023}. CGC is important in the coastal environment because
it can cause various coastal hazards. Two of the most common coastal hazards
caused by CGC are bluff failure and shoreline retreat (see Figure
\ref{fig:fig1.2}), both of which are critically important in coastal
environments due to their significant impacts on human infrastructure and
ecosystems. Bluff erosion at both the crest and toe of bluff could undermine the
stability of bluff structures, potentially leading to a bluff failure. Bluff
failure, including collapses, landslides, and slumps, as depicted in Figure
\ref{fig:fig1.2}a, is particularly destructive, as it can damage properties and
infrastructures built at the top or base of the bluff, leading to economic
losses and even casualty \citep{deitz_bluff_2024}. For instance, between 2019
and 2022, there were three reported news
\citep{mitchell_geologist_2019,fromson_house_2020,koran_15_2022} of homes
collapsing due to unstable coastal bluffs in Lake Michigan. Another common issue
is shoreline retreat, which is well-known for its adverse effects on coastal and
nearshore ecosystems, as well as biodiversity. For instance, dune ecosystems,
which provide dynamic habitats for species such as beach grass, sand verbena,
and plovers, can be severely degraded by shoreline retreat
\citep{van_der_biest_dune_2017}. Recently, CGC has become an urgent issue, as
accelerated erosion along coastal bluffs and shorelines has been observed,
reaching its highest rate in the past 30 years
\citep{troy_rapid_2021,zoet_analysis_2017,gronewold_hydrological_2016}. Facing
threats from these two issues, numerous efforts have been made. For example, in
southeastern Wisconsin, over $2.89$ billion was spent to construct, repair and
maintain the structures to protect the local properties and infrastructure
\citep{sanders_2016_nodate}. Despite these considerable expenditures to
mitigate the risks of CGC, there remains an additional critical step: coastal
geomorphology mappings on multiple temporal and spatial scales. Multi-scale
coastal mapping can provide insights into the evolution of coastal geomorphology
across varying temporal and spatial scales, offering valuable guidance and
perspectives for effective prediction and prevention for coastal hazards
\citep{kolednik2014coastal,papakonstantinou2016coastline}. In Lake Michigan,
although numerous projects and studies have been made for characterization of
coastal geomorphic change, most have been limited to small regions
\citep[\eg][]{swenson_bluff_2006,kilibarda_70year_2015} or short time periods
\citep[\eg][]{lin_field_2014,theuerkauf_coastal_2019}, leaving multi-spatial and
multi-temporal CGC insufficiently explored. Given the lack of a clear
characterization of CGC, it would be valuable to conduct an extensive mapping
and analysis across the coastal region of southwestern Lake Michigan to address
gaps in understanding CGC over multi-spatial and multi-temporal scales.

\begin{figure}[htbp]
  \centering
  \includegraphics[width=0.8\textwidth]{chapter1/resources/figure1-2.jpg}
  \caption{images that show two typical hazards caused by CGC: (a) bluff failure, (b) shoreline retreat.}
  \label{fig:fig1.2}
\end{figure}


\subsection{Climatology of water level and wind wave}
\label{subsec:Climatology of water level and wind wave}

There are two important hydrodynamic processes that significantly affect coastal
geomorphology: water level and wave. The climatology of water level, which
refers to the study of the elevation fluctuation of lake or sea surface, is the
most well-studied topic in the coastal region. The climatology of water level is
important as it implies the stability of coastline, health of coastal habitat
\citep{theuerkauf_rapid_2021}, and safety of ship navigation
\citep{posey2012climate}, etc. To better understand the climatology of
hydrodynamic process, numerous efforts have been made. For example, in the Great
Lakes, the records of water levels were initially documented since the 1820s
\citep{foster_report_1851} and have been systematically monitored across the
lakes since the 1920s by NOAA. In Lake Michigan, NOAA has deployed 10 water
level gauges since the 1970s to monitor monthly lake level fluctuations with
verification, as shown in Figure \ref{fig:fig1.3}. These long-term data have
allowed for a comprehensive understanding of lake level climatology, including
inter-annual trend \citep{hanrahan_attribution_2014,chen_understanding_2022},
seasonal cycles \citep{argyilan_lake_2003,quinn_secular_2002}, and daily events
\citep{trebitz_characterizing_2006}. Furthermore, the mechanisms driving lake
level fluctuations have been widely explored, including factors such as
precipitation \citep{rodionov_association_1994,hanrahan_attribution_2014},
runoff and evaporation
\citep{gronewold_tug--war_2021,gronewold_hydrological_2016,cheng_effects_2021},
ice cover \citep{farhadzadeh_study_2017}, and atmospheric teleconnections
\citep{ghanbari_coherence_2008}.

\begin{figure}[htbp]
  \centering
  \includegraphics[width=0.8\textwidth]{chapter1/resources/figure1-3.jpg}
  \caption{Monthly water level fluctuation in Lake Michigan from NOAA tide\&current.}
  \label{fig:fig1.3}
\end{figure}

Wind wave, a periodic circular motion of fluid driven by wind force, is another
important nearshore process, as it is closely related to water level fluctuation
\citep{meadows_relationship_1997,huang_impacts_2021,huang_wave_2021}, coastal
structure constructions \citep{karambas_modelling_2015}, and coastal energy
generation \citep{ching-piao_study_2012}. Similar to water level, wave
climatology is key to understanding the hydrodynamic property of wave processes.
Nevertheless, compared to the climatology of water level, which primarily
focuses on magnitude of lake elevation, the wind wave climate involves a broader
range of wave characteristics such as wave directionality and wave spectrum, due
to the more complex nature of wave processes. For example, waves can be bent
perpendicular to the contours of the bathymetry as wave propagates in different
water depths, which is known as the wave refraction. The wave refraction often
results in uni-directional wave climate on the coast where all waves are bent
perpendicular to the shoreline (see Figure \ref{fig:fig1.4}a). Conversely, when
waves encounter obstacles like groins or breakwaters, they can bend into a
bi-directional pattern, traveling in opposite or oblique directions due to wave
diffraction (see Figure \ref{fig:fig1.4}b). Both uni-directional and
bi-directional wave climates are commonly observed in Lake Michigan (see Figure
\ref{fig:fig1.4}c), and are important for navigation safety
\citep{zimmermann_longshore_2012}, wave energy farm construction
\citep{lopez-ruiz_importance_2016,bertram_systematic_2020}, and beach rotation
\citep{wiggins_coastal_2019,wiggins_regionally-coherent_2019}. Another important
wave process is wave systems. Wind waves can be classified into two wave
systems: wind-sea and swell, which represent different wave states. Wind-sea
refers to waves generated by local winds, typically characterized by longer wave
periods and less developed waveforms. In contrast, swell refers to waves that
have grown and propagated beyond the influence of the wind that originally
generated them. Swells can travel long distances with minimal energy loss and
are distinguished by their shorter wave period \citep{ardhuin_observation_2009}.
In wave climatology, windsea and swell are regarded as two components of the
spectral wave climate, and spectral wave climate is important for wind wave data
reduction, model validation, and wave event tracing
\citep{portilla-yandun_wave_2015}. In summary, wind wave climate involves more
complex wave characteristics such as wave height, wave energy, wave direction,
wave spectrum. In Great Lakes, most studies of wave climates have focused on the
annual or monthly trend of wave height
\citep[\eg][]{olsen_long_2019,huang_impacts_2021,huang_wave_2021} or wave energy
\citep[\eg][]{meadows_relationship_1997}. To date, the directional and spectral
wave climate in the Great Lakes has not yet been revealed. Hence, further
exploring to understand the directional spectral wave climate in Lake Michigan
may shed light on complex hydrodynamic processes.

\begin{figure}[htbp]
  \centering
  \includegraphics[width=0.8\textwidth]{chapter1/resources/figure1-4.jpg}
  \caption{The photos of directional wave climate: (a) uni-directional and (b) bi-directional, associated with (c)the maps of the directional wave climate in Lake Michigan.}
  \label{fig:fig1.4}
\end{figure}

\subsection{Relationship between coastal geomorphology and wave climate}
\label{Relationship between coastal geomorphology and wave climate}

It is well-known that water level could significantly influence the coastal
environment in Great Lakes, including wetland habitat
\citep{hohman_influence_2021,anderson_influence_2023}, sandy dunes
\citep{arbogast_maximum-limiting_1999,kilibarda_70year_2015}, bluff
\citep{volpano_three-dimensional_2020} and beach \citep{scholle_responses_2022}.
Nevertheless, wave climates, especially wave height, directionality, and wave
systems, are sometime ignored for its huge impact on coastal geomorphologies
including shoreline and bluff. For example, wave height determines the wave
energy flux, which is regarded as an important indicator as coastal bluff
erosion and shoreline recession
\citep{benumof_relationship_2000,galal_influence_2011}. Wave height can be
further amplified through wave shoaling effect (Figure \ref{fig:fig1.5}a) and
refraction effect (Figure \ref{fig:fig1.5}b), leading to a steeper lakebed and
bluff face, which causes accelerated erosion in the shoreline and failure in
bluff \citep{booth_wave_1994}. Moreover, the wave-driven nearshore sediment
transport can also lead to the redistribution of nearshore sediment budgets,
which is essential to shoreline erosion
\citep{amin_statistical_1997,usace_cem_2002,adams_effects_2011,dean2004coastal}.
Cumulative wave impact height, which is an index derived from the nearshore wave
height and water level, was also found to be a control of the coastal erosion
\citep{ruggiero_wave_2001,swenson_bluff_2006}. Wave directionality and wave
systems are also found to be highly related to coastal geomorphology. For
example, numerous studies found that bidirectional wave climate, a special wave
directionality with two opposite wave directions, can cause coastal beach
rotation, which is a common coastal process occurred at many semi-sheltered and
embayed shorelines
\citep{klein_short-term_2002,wiggins_coastal_2019,wiggins_regionally-coherent_2019,loureiro_24_2020}.
However, despite of the fact that wave climate could impact coastal geomorphic
changes, the relationship between coastal geomorphic change and wind wave
climate, particularly directionality and wave systems, is still under-studied
yet. 

\begin{figure}[htbp]
  \centering
  \includegraphics[width=0.8\textwidth]{chapter1/resources/figure1-5.jpg}
  \caption{wave climates in the coastal environment: (a) wind wave including windsea and swell under shoaling effect, (b) refraction and diffraction.}
  \label{fig:fig1.5}
\end{figure}


\section{Research questions and objectives}

To date, the importance of coastal geomorphic changes and wind wave climate has
been widely recognized in the world. However, in Lake Michigan, several
knowledge gaps persist, limiting our understanding of these critical processes,
which leads to the formulation of the following research questions:

\begin{enumerate}
    \item Coastal geomorphology: What are the multiple scales of coastal geomorphic changes on Lake Michigan?
    \item Wind wave climate: What are the characteristics of wind wave climate, especially directionality and wave systems in Lake Michigan?
    \item Relationships: How to indicate the CGC with the assistant of directional spectral wave climate?
\end{enumerate}

In response to these three unanswered questions, the proposed research
objectives are: (1) to characterize coastal geomorphological changes and wind
wave climate in Lake Michigan, and (2) to investigate the relationship between
them. This research is structured into four chapters, each addressing critical
gaps to answer the overarching research questions regarding coastal
geomorphological changes and wind wave climates. The specific research question,
objective, and contribution for each chapter are summarized in the following
table:

\renewcommand{\arraystretch}{1.4}

\begin{longtable}{|>{\raggedright\arraybackslash}p{2.4cm}|p{12cm}|}
\caption{Chapter Overview} \\
\hline
\multicolumn{2}{|c|}{\textbf{Chapter 2}} \\
\hline
\textbf{Topic} & \textit{Coastal geomorphological changes on southwestern Lake Michigan: Perspective from multi-scale} \\
\textbf{Question} & What are the coastal geomorphological changes on multi-scale in the coastline of southwestern Lake Michigan? \\
\textbf{Objective} & To characterize coastal geomorphological changes on different temporal and spatial scales. \\
\textbf{Contribution} & Provide an insight of bluff and shoreline change over both long-term and short-term periods, utilizing three distinct spatial scales. \\
\hline
\multicolumn{2}{|c|}{\textbf{Chapter 3}} \\
\hline
\textbf{Topic} & \textit{Wave climate on the southeastern Wisconsin coast of Lake Michigan: Perspective from wave directionality} \\
\textbf{Question} & Does wave directionality help identify the trend of wave climate? \\
\textbf{Objective} & To characterize the directional wave climate in the nearshore of southeastern Lake Michigan. \\
\textbf{Contribution} & Provide an approach to define the wave directionality and explore the implication of wave directionality. \\
\hline
\multicolumn{2}{|c|}{\textbf{Chapter 4}} \\
\hline
\textbf{Topic} & \textit{Wave climate on the southeastern Wisconsin coast of Lake Michigan: Perspective from wave systems} \\
\textbf{Question} & Does wave spectrum help identify the trend of wave climate? \\
\textbf{Objective} & To analyze the effect of swells and wind–sea waves at the nearshore of Lake Michigan. \\
\textbf{Contribution} & Illustrate the trend of wave climate by swell–wind–sea separation and by probability distribution in spectral occurrence map. \\
\hline
\multicolumn{2}{|c|}{\textbf{Chapter 5}} \\
\hline
\textbf{Topic} & \textit{Assessing coastal vulnerability using directional spectral wave climate: Coastal Vulnerability Index} \\
\textbf{Question} & How does the directional spectral wave climate impact the coastal vulnerability? \\
\textbf{Objective} & Develop an index to access the coastal vulnerability that is applicable to the directional spectral wave climate in Lake Michigan. \\
\textbf{Contribution} & Provide an insight in coastal vulnerability in Lake Michigan coastline. \\
\hline
\end{longtable}


In summary, this study aims to advance the understanding of coastal
geomorphology and wind wave climate, as well as the intricate relationship
between them.
% Your chapter content goes here...
