\begin{table}[ht]
  \centering
  \footnotesize
  \setlength{\tabcolsep}{4pt} % tighter spacing between columns
  \renewcommand{\arraystretch}{1.2} % row height
  \begin{tabularx}{\textwidth}{l l l X p{2cm}}
    \toprule
    \textbf{Date} & \textbf{Photo Source} & \textbf{Spatial Extent} & \textbf{Photo Description} & \textbf{Photo Scale/Resolution (m)} \\
    \midrule
    August 1937   & USDA            & Entire Study Area & B\&W Aerial Photo       & 1--1.6 \\
    May--July 1956 & USDA           & Entire Study Area & B\&W Aerial Photo       & 1 \\
    June 1969     & USDA            & Racine Co.        & B\&W Aerial Photo       & 1 \\
    August 1971   & USDA            & Ozaukee Co.       & B\&W Aerial Photo       & 1.8 \\
    April 1975    & Milwaukee Cty LIO & Milwaukee Co.   & B\&W Aerial Photo       & 0.2 \\
    March 1976    & USGS            & Kenosha Co.       & B\&W Aerial Orthophoto  & 3 \\
    April 1995    & SEWRPC          & Entire Study Area & B\&W Aerial Orthophoto  & 0.61 \\
    April 2000    & SEWRPC          & Entire Study Area & B\&W Aerial Orthophoto  & 0.31 \\
    April 2005    & SEWRPC          & Entire Study Area & Color Aerial Orthophoto & 0.31 \\
    April 2010    & SEWRPC          & Entire Study Area & Color Aerial Orthophoto & 0.31 \\
    April 2015    & SEWRPC          & Entire Study Area & Color Aerial Orthophoto & 0.15 \\
    April 2020    & SEWRPC          & Entire Study Area & Color Aerial Orthophoto & 0.15 \\
    \bottomrule
  \end{tabularx}
  \caption{Aerial photo information for photos used in the analysis.}
  \label{tab:tab2.2}
\end{table}
