\begin{table}[ht]
  \centering
  \small
  \setlength{\tabcolsep}{4pt}
  \renewcommand{\arraystretch}{1.2}
  \begin{tabularx}{\textwidth}{l *{3}{>{\centering\arraybackslash}p{2.2cm}} *{3}{>{\centering\arraybackslash}p{2.2cm}}}
    \toprule
    \multicolumn{1}{c}{\textbf{County}} &
    \multicolumn{3}{c}{\textbf{1937--2020 Recession Rate (m/yr)}} &
    \multicolumn{3}{c}{\textbf{1995--2020 Recession Rate (m/yr)}} \\
    \cmidrule(lr){2-4}\cmidrule(lr){5-7}
    & \textbf{Bluff Crest} & \textbf{Bluff Toe} & \textbf{Shoreline}
    & \textbf{Bluff Crest} & \textbf{Bluff Toe} & \textbf{Shoreline} \\
    \midrule
    \textit{Ozaukee}   & -0.24 & -0.21 & -0.21 & -0.26 & -0.15 & -0.39 \\
    \textit{Milwaukee} & -0.18 & -0.05 & -0.06 & -0.22 & -0.11 & -0.26 \\
    \textit{Racine}    & -0.30 & -0.24 & -0.16 & -0.19 & -0.24 & -0.34 \\
    \textit{Kenosha}   & -0.17 & -0.28 & -0.40 & -0.14 & -0.11 & -0.24 \\
    Mean               & -0.22 & -0.17 & -0.16 & -0.22 & -0.15 & -0.32 \\
    \bottomrule
  \end{tabularx}
  \caption{County-averaged bluff crest, bluff toe, and shoreline recession rates. Negative values indicate bluff erosion, while positive values indicate bluff advance.}
  \label{tab:tab2.3}
\end{table}
