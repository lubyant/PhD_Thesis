\begin{table}[h!]
\caption{Reach-averaged bluff slope and beach width for the 1937--2020 and 1995--2020 periods. Negative values indicate erosion, and positive values indicate lakeward movement of the shoreline. Note that Reach 1 and Reach 13 do not contain cohesive bluffs, thus no beach width is presented as no backshore feature was delineated.}
\centering
\renewcommand{\arraystretch}{1.2}
\begin{tabularx}{\textwidth}{c *{2}{>{\centering\arraybackslash}X} *{2}{>{\centering\arraybackslash}X}}
\hline
\multirow{2}{*}{Reach ID} & 
\multicolumn{2}{c}{\textbf{Long-term average (1937--2020)}} & 
\multicolumn{2}{c}{\textbf{Short-term average (1995--2020)}} \\
\cline{2-5}
& Beach width & Bluff face slope & Beach width & Bluff face slope \\
\hline
1  & --   & --   & --    & --   \\
2  & 19.80 & 0.54 & 18.48 & 0.53 \\
3  &  6.18 & 0.52 &  7.13 & 0.51 \\
4  &  7.09 & 0.63 &  7.58 & 0.64 \\
5  & 12.69 & 0.42 & 13.09 & 0.42 \\
6  & 10.68 & 0.40 &  9.79 & 0.38 \\
7  & 18.62 & 0.52 & 20.29 & 0.51 \\
8  & 14.18 & 0.48 & 15.46 & 0.45 \\
9  &  5.13 & 0.62 &  4.91 & 0.60 \\
10 & 12.40 & 0.43 & 12.92 & 0.39 \\
11 & 65.51 & 0.38 & 76.64 & 0.37 \\
12 &  9.06 & 0.43 &  8.16 & 0.47 \\
13 & --   & --   & --    & --   \\
\hline
\end{tabularx}
\label{tab:tab2.6}
\end{table}
