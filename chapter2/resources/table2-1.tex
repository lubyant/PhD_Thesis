\begin{table}[htbp]
  \caption{Definitions of coastal reaches.}
  \centering
  \setlength{\tabcolsep}{8pt}
  \renewcommand{\arraystretch}{1.15}
  \begin{tabularx}{\textwidth}{c X}
    \toprule
    \textbf{Reach ID} & \textbf{Description} \\
    \midrule
    1  & Flat, sandy, no bluff, heavily developed \\
    2  & Low bluff, heavily developed \\
    3  & Flat coast/very low bluff, sandy, mostly defended, medium development, SE orientation \\
    4  & Low bluff, heavily defended, sandy sediments, NE orientation \\
    5  & High cohesive bluff, little to no shore protection \\
    6  & High cohesive bluff, industrial land uses, increasingly defended since 1937 \\
    7  & High cohesive bluff with few structures, primarily recreational land use \\
    8  & High bluff, heavily developed residential land use \\
    9  & Donges Bay, high bluff, moderately defended, moderate developed land use \\
    10 & High bluff, moderate bluff top development and shore defense \\
    11 & High bluff, sparse bluff top development \\
    12 & High bluff, sparse bluff top development \\
    13 & Low dune/flat coast, no bluff \\
    \bottomrule
  \end{tabularx}
  \label{tab:tab2.1}
\end{table}
