Coastal geomorphological change, including the erosion of coastal bluffs and beaches, is a prevalent coastal hazard that poses significant threats to people and property in coastal areas. Recently, concerns over bluff and beach erosion have intensified due to the rapid rise in water levels in Lake Michigan. To mitigate the impending risks associated with these geomorphological changes, a critical question arises: how can coastal geomorphological changes be linked to coastal hydrodynamic processes such as wave climates and water level climates? To address this question, we hypothesize that coastal geomorphological changes are driven by a combination of multiple factors, including water level fluctuations, wave climate dynamics, and nearshore sediment transport processes. To examine this hypothesis, we first characterized historical coastal geomorphological changes in southwestern Lake Michigan using aerial photographs. These changes were further analyzed across multiple spatial and temporal scales, providing a foundational dataset for the subsequent phases of the study. Second, the directionality of wave climate in Lake Michigan was characterized. The bidirectionality, a special form of wave directionality, was identified in Lake Michigan and characterized by its temporal trend and spatial variability. Third, the spectral wave climate was characterized in Lake Michigan to reflect the wave systems of Lake Michigan. Wave systems, including swell and wind-sea waves, were discussed for the directional and spectral characteristics. Finally, to integrate the findings from the previous three sections, an innovative approach for assessing coastal geomorphological changes—the Coastal Vulnerability Index (CVI)—was developed. This index was formulated based on a combination of factors, including wave directionality, wave systems, water levels, and longshore sediment transport. Overall, this research seeks to enhance the understanding of the interactions between coastal geomorphology and Lake Michigan’s unique climatological and hydrodynamic conditions.